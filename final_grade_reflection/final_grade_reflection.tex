% Options for packages loaded elsewhere
\PassOptionsToPackage{unicode}{hyperref}
\PassOptionsToPackage{hyphens}{url}
\PassOptionsToPackage{dvipsnames,svgnames,x11names}{xcolor}
%
\documentclass[
  letterpaper,
  DIV=11,
  numbers=noendperiod]{scrartcl}

\usepackage{amsmath,amssymb}
\usepackage{lmodern}
\usepackage{iftex}
\ifPDFTeX
  \usepackage[T1]{fontenc}
  \usepackage[utf8]{inputenc}
  \usepackage{textcomp} % provide euro and other symbols
\else % if luatex or xetex
  \usepackage{unicode-math}
  \defaultfontfeatures{Scale=MatchLowercase}
  \defaultfontfeatures[\rmfamily]{Ligatures=TeX,Scale=1}
\fi
% Use upquote if available, for straight quotes in verbatim environments
\IfFileExists{upquote.sty}{\usepackage{upquote}}{}
\IfFileExists{microtype.sty}{% use microtype if available
  \usepackage[]{microtype}
  \UseMicrotypeSet[protrusion]{basicmath} % disable protrusion for tt fonts
}{}
\makeatletter
\@ifundefined{KOMAClassName}{% if non-KOMA class
  \IfFileExists{parskip.sty}{%
    \usepackage{parskip}
  }{% else
    \setlength{\parindent}{0pt}
    \setlength{\parskip}{6pt plus 2pt minus 1pt}}
}{% if KOMA class
  \KOMAoptions{parskip=half}}
\makeatother
\usepackage{xcolor}
\setlength{\emergencystretch}{3em} % prevent overfull lines
\setcounter{secnumdepth}{-\maxdimen} % remove section numbering
% Make \paragraph and \subparagraph free-standing
\ifx\paragraph\undefined\else
  \let\oldparagraph\paragraph
  \renewcommand{\paragraph}[1]{\oldparagraph{#1}\mbox{}}
\fi
\ifx\subparagraph\undefined\else
  \let\oldsubparagraph\subparagraph
  \renewcommand{\subparagraph}[1]{\oldsubparagraph{#1}\mbox{}}
\fi


\providecommand{\tightlist}{%
  \setlength{\itemsep}{0pt}\setlength{\parskip}{0pt}}\usepackage{longtable,booktabs,array}
\usepackage{calc} % for calculating minipage widths
% Correct order of tables after \paragraph or \subparagraph
\usepackage{etoolbox}
\makeatletter
\patchcmd\longtable{\par}{\if@noskipsec\mbox{}\fi\par}{}{}
\makeatother
% Allow footnotes in longtable head/foot
\IfFileExists{footnotehyper.sty}{\usepackage{footnotehyper}}{\usepackage{footnote}}
\makesavenoteenv{longtable}
\usepackage{graphicx}
\makeatletter
\def\maxwidth{\ifdim\Gin@nat@width>\linewidth\linewidth\else\Gin@nat@width\fi}
\def\maxheight{\ifdim\Gin@nat@height>\textheight\textheight\else\Gin@nat@height\fi}
\makeatother
% Scale images if necessary, so that they will not overflow the page
% margins by default, and it is still possible to overwrite the defaults
% using explicit options in \includegraphics[width, height, ...]{}
\setkeys{Gin}{width=\maxwidth,height=\maxheight,keepaspectratio}
% Set default figure placement to htbp
\makeatletter
\def\fps@figure{htbp}
\makeatother

\KOMAoption{captions}{tableheading}
\makeatletter
\makeatother
\makeatletter
\makeatother
\makeatletter
\@ifpackageloaded{caption}{}{\usepackage{caption}}
\AtBeginDocument{%
\ifdefined\contentsname
  \renewcommand*\contentsname{Table of contents}
\else
  \newcommand\contentsname{Table of contents}
\fi
\ifdefined\listfigurename
  \renewcommand*\listfigurename{List of Figures}
\else
  \newcommand\listfigurename{List of Figures}
\fi
\ifdefined\listtablename
  \renewcommand*\listtablename{List of Tables}
\else
  \newcommand\listtablename{List of Tables}
\fi
\ifdefined\figurename
  \renewcommand*\figurename{Figure}
\else
  \newcommand\figurename{Figure}
\fi
\ifdefined\tablename
  \renewcommand*\tablename{Table}
\else
  \newcommand\tablename{Table}
\fi
}
\@ifpackageloaded{float}{}{\usepackage{float}}
\floatstyle{ruled}
\@ifundefined{c@chapter}{\newfloat{codelisting}{h}{lop}}{\newfloat{codelisting}{h}{lop}[chapter]}
\floatname{codelisting}{Listing}
\newcommand*\listoflistings{\listof{codelisting}{List of Listings}}
\makeatother
\makeatletter
\@ifpackageloaded{caption}{}{\usepackage{caption}}
\@ifpackageloaded{subcaption}{}{\usepackage{subcaption}}
\makeatother
\makeatletter
\@ifpackageloaded{tcolorbox}{}{\usepackage[many]{tcolorbox}}
\makeatother
\makeatletter
\@ifundefined{shadecolor}{\definecolor{shadecolor}{rgb}{.97, .97, .97}}
\makeatother
\makeatletter
\makeatother
\ifLuaTeX
  \usepackage{selnolig}  % disable illegal ligatures
\fi
\IfFileExists{bookmark.sty}{\usepackage{bookmark}}{\usepackage{hyperref}}
\IfFileExists{xurl.sty}{\usepackage{xurl}}{} % add URL line breaks if available
\urlstyle{same} % disable monospaced font for URLs
\hypersetup{
  pdftitle={Final Grade Reflection},
  pdfauthor={Fiona Norton},
  colorlinks=true,
  linkcolor={blue},
  filecolor={Maroon},
  citecolor={Blue},
  urlcolor={Blue},
  pdfcreator={LaTeX via pandoc}}

\title{Final Grade Reflection}
\author{Fiona Norton}
\date{}

\begin{document}
\maketitle
\ifdefined\Shaded\renewenvironment{Shaded}{\begin{tcolorbox}[boxrule=0pt, frame hidden, interior hidden, breakable, enhanced, sharp corners, borderline west={3pt}{0pt}{shadecolor}]}{\end{tcolorbox}}\fi

In this document, you make a data-based argument for the grade you've
earned in this course. Your argument should include evidence from the
supporting artifacts you've provided.

The output document should be a PDF or a Word Document, as it should be
a \textbf{maximum} of 2-pages.

\hypertarget{week-6-reflection}{%
\section{Week 6 Reflection}\label{week-6-reflection}}

\hypertarget{accomplishment-of-learning-targets}{%
\subsection{Accomplishment of Learning
Targets}\label{accomplishment-of-learning-targets}}

Over the past six weeks I have strengthened my knowledge of R and R
Studio through each preview activity, practice activity, lab assignment,
and challenge problem. I have shown my proficiency in the majority of
the learning targets and shown my desire to do better in areas that I do
not feel as strong in. Beginning with the ``Working with data learning
targets, I can import data from a variety of formats which can be seen
at the beginning of almost every assignment in the''setup'' section
(specifically labs 1-5). I can select necessary columns from a dataset,
filter rows from a dataframe, and modify existing variables and create
new variables in a dataframe for a variety of data types, as shown in my
week 3 practice activity, my week 5 practice activity, and question 6 on
lab 3. I can use mutating joins to combine multiple dataframes and
filtering joins to filter rows from a dataframe; this can be seen in lab
4 questions 2, 5, and 6.

I am also proficient in data visualization and summarization, and I
believe I have made a lot of progress in this area since my week 3
reflection. I can create visualizations for a variety of variable types,
I use plot modifications to make my visualization clear to the reader,
and I show creativity in my visualizations. All of these skills can be
seen in my plots on recent labs, specifically lab 4 questions 6 and 7,
lab 2 questions 9, 10, and 12, and almost all of lab 5. I've shown that
I can calculate numerical summaries of variables in lab 4 question 6 and
the ``Familiar Words'' section of lab 3. I can find summaries of
variables across multiple groups as I showed in challenge thre parts 2
and 3. I can create tables which make my summaries clear to the reader
like in lab 4 question 5 (finding top 5 regions), and challenge 3 parts
2 and 3.

Finally I have shown proficiency in reproducibility and efficiency of my
code. I have shown that I can create a reproducible analysis using
RStudio projects, Quarto documents, and the here package at the
beginning of each lab by importing the data with the ``here'' package
and including the correct libraries. I can write well documented and
tidy code winch I try to always do but can be seen specifically in lab 4
question 4. I can use iteration to reduce repetition in my code, like in
challenge 3 parts 2 and 3.

\hypertarget{evidence-of-continued-learning}{%
\subsection{Evidence of Continued
Learning}\label{evidence-of-continued-learning}}

\hypertarget{extending-my-thinking}{%
\subsubsection{Extending My Thinking}\label{extending-my-thinking}}

Each week I show my willingness and ability to extend my thinking
through the completion of challenge problems. In all of my challenge
problems I dive deeper into a specific aspect of what we are learning
that week ad try to further my understanding of it. One example of this
is the customization/challenge section of lab 2 where I attempted to
create and easier to read side by side box plot of the weights of
different rodents by species including a color and label for the genus
of that species as well as a customized color pallet. I am proficient in
the learning targets discussed in the previous section because of the
ways in which I am extending my thinking.

\hypertarget{revising-my-thinking}{%
\subsubsection{Revising My Thinking}\label{revising-my-thinking}}

I am constantly revising my thinking by going back to old assignments
and making improvements and working through preview activities multiple
times to try to gain a better understanding of topics before coming to
class. Each week I submit revisions to my labs and challenges based on
the feedback provided by professor Theobold and my peers as well as the
new knowledge I have gained in the time since the original submission.
Along with these revisions, I write reflections about what I have
learned from the changes and how I will use that knowledge moving
forward. Good examples of how I have revised and extended my thinking
include my lab 3 and lab 4 revision reflections. The revision process
has helped me learn much more about R than I would have without
receiving any feedback.

\hypertarget{growth-as-a-team-member}{%
\subsection{Growth as a Team Member}\label{growth-as-a-team-member}}

\hypertarget{collaborative-group-work}{%
\subsubsection{Collaborative Group
Work}\label{collaborative-group-work}}

My growth as a team member throughout this course has mainly come as a
result of our in class collaborative group work. Every Tuesday we work
on a practice activity as a team. Our group has created an environment
where everyone feels safe to ask questions and speak their minds. OUr
completion of these practice activities is a reflection of how well we
are working together. Although we do not explicitly assign roles each
week, there is a natural rotation of the roles depending on who is
feeling strongest in the topic for that week. That person often emerges
as the leader/captain for the week and the rest of us are happy to fall
into other roles. I have realized that it can be difficult for me to ask
questions when I feel behind because I do not want to hold back the
group but I have come to understand the importance of working as a team
and realize that explaining things to other people is helpful for
everyone's learning so it is good to ask questions.

\hypertarget{peer-code-review}{%
\subsubsection{Peer Code Review}\label{peer-code-review}}

I have completed each assigned peer code review carefully and with a lot
of thought. I understand the importance of being kind to my peers but
also realize that feedback is helpful so I try to give in depth reviews
with words of encouragement. This can be seen in my peer reviews
throughout the quarter.

\hypertarget{attention-to-personal-goals}{%
\subsection{Attention to Personal
Goals}\label{attention-to-personal-goals}}

At the beginning of this course I thought about my personal goals and
decided to focus on improving my data visualization skills. I feel that
I've made drastic progress in this area over the past 6 weeks and I hope
to continue doing so. I have learned how to add colors and labels to a
graph, how to create many different kinds of graphs with ggplot, how to
customize a legend or exclude it all together, and how to graph only
certain specific aspects of the data by combining the use of ggplot with
dplyer functions. Additionally I wanted a general sense of ``knowing
R''. Although this is very difficult to measure, I do feel like I am
getting closer and closer each week.



\end{document}
